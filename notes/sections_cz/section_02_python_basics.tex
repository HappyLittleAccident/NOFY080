% Sekce 2: Základy Pythonu
\newpage
\section{Základy Pythonu}

\subsection{Proměnné a typy}

Proměnné se vytvářejí přiřazením hodnoty k dosud nepoužitému jménu proměnné. Deklarace, jako v C/C++ nebo Pascalu, nejsou nutné.
\begin{lstlisting}[caption=Definování proměnných]
    my_integer = 5
    my_float = 3.14
    my_string = "double quote string"
    my_string2 = 'single quote string'
    my_string3 = "you can use 'single' quote in double quote string"
    my_bool = False
\end{lstlisting}

Všimněte si, že proměnné obecně nemají pevný typ (tj. celé číslo nebo řetězec) a provedení něčeho jako \lstinline{my_integer = "a string"} nezpůsobí typovou chybu. Samotný Python se nestará o to, jakého typu proměnná je, pokud jsou operace, které s ní provádíme, podporovány. Pokud však potřebujeme znát typ proměnné \lstinline{v}, můžeme jej zjistit pomocí \lstinline{type(v)}.

Základní aritmetické operace \verb|+,-,*,/| fungují na číslech podle očekávání. Umocňování je \verb|**|. Všimněte si však, že \verb|/| automaticky \emph{povýší (promotion)} celá čísla na desetinná (float). Pro celočíselné dělení můžeme použít \verb|//| a pro zbytek po dělení \verb|%|. Aritmetické operace jsou také definovány pro některé nečíselné typy, kde "dávají smysl", tj. řetězce lze spojovat pomocí \verb|+|. Porovnávací operátory jsou \verb|==| pro rovnost (\textbf{nezaměňovat s =, což je přiřazení hodnoty proměnné}), \verb|!=| pro nerovnost a \verb|<,<=,>,>=| pro uspořádání. Booleovské operátory jsou \verb|and, or, not|.

Existují také \textbf{bitové} operátory \lstinline{&,|,~,^} (bitové AND, OR, NOT a XOR). Tyto operují na jednotlivých bitech čísla. Dejte si pozor na záměnu mezi umocňováním \ls{**} a bitovým exkluzivním \emph{or} (XOR) \ls{^}.

Komplexní čísla jsou v Pythonu podporována. Syntaxe pro zadání komplexního čísla je např. \ls{3 + 5j}, což se překládá do matematického zápisu $3 + 5i$. Konkrétně imaginární jednotka je \ls{1j}. Všimněte si, že písmeno \ls{j} následuje bezprostředně za číslem.

\begin{exercise}
    Vyzkoušejte si základní aritmetiku v REPL. Zkuste vynásobit řetězec celým číslem. Co se stane, když se ho pokusíte vydělit?
\end{exercise}

Shrnutí základních vestavěných operátorů

\begin{tabular}{ll}
     \ls{+,-,*,/, **} & základní aritmetika, umocňování \\ 
     \ls{and, or, not} & booleovské operace \\
     \ls{<, <=, ==, >, >=} & porovnávání\\
     \ls{\&,|, ~, ^} & bitové operace, bitové AND, OR, NOT a XOR (exkluzivní nebo)
\end{tabular}

\subsection{Uživatelský vstup}
Většinou budeme psát neinteraktivní skripty, nicméně pro získání základního textového vstupu od uživatele můžeme použít funkci \verb|input()|, \textbf{která vždy vrací řetězec}. Musíme jej sami převést na číslo (pomocí \verb|int()| nebo \verb|float()|), pokud očekáváme, že uživatel zadá číslo, např.:
\begin{lstlisting}
    age = int(input("How old are you? "))
    print("You will be 100 in", 100-age, "years.")
\end{lstlisting}

\subsection{Výpis a řetězce}
Již jsme se setkali s funkcí \verb|print()|, která na standardní výstup vypíše textovou reprezentaci objektů, které jí předáme. Přijímá jeden nebo více argumentů.

Pro převod jakékoli hodnoty na řetězec můžeme použít funkci \verb|str()|. Speciální znaky, jako je nový řádek nebo tabulátor, je třeba \emph{escapovat}, např. \verb|\n| a \verb|\t|. Zpětná lomítka je také třeba escapovat \verb|\\|. Abychom zabránili Pythonu v interpretaci zpětných lomítek jako speciálních znaků (užitečné např. pro zadávání cest k souborům ve Windows), můžeme použít tzv. raw řetězce, např.:
\begin{lstlisting}
    print("s\tri\ng") #\t and \n are interpreted as tab and new line
    print(r"s\tri\ng") # raw string, note the r immediately before "
\end{lstlisting}

Pro vytváření složitějších výpisů našich proměnných můžeme použít \emph{formátovací} řetězce (nebo f-stringy) tak, že proměnnou, kterou chceme vložit do řetězce, uzavřeme do \verb|{}|, např.:
\begin{lstlisting}
    vari = 5
    varf = 3.14
    fstr = f"An interger value {vari:05d}, a float value {varf:010.6f}, exponential form {varf:g}"
\end{lstlisting}
Nahrazení textu hodnotou proměnné (tzv. interpolace) se specifikuje jako\\\verb|{výraz:specifikátor_formátu}|,\\ přičemž specifikátory formátu jsou volitelné. Zde specifikátor formátu \verb|05d| znamená celé číslo s celkovou šířkou alespoň 5 znaků a \verb|010.6f| znamená číslo s plovoucí desetinnou čárkou se 6 číslicemi za desetinnou čárkou a celkovou šířkou alespoň 10 znaků.

Pro úplnou specifikaci možných formátovacích specifikátorů viz \href{https://docs.python.org/3/library/string.html\#formatspec}{dokumentaci}.

\subsection{Funkce}
Funkce se definují pomocí klíčového slova \verb|def|. Funkce mohou přijímat argumenty a vracet hodnotu.
\begin{lstlisting}
    def my_function(my_string):
        print("Someone wants to know how long \"" + my_string + "\" is!")
        return len(my_string) #returns the length of a my_string
\end{lstlisting}
Všimněte si odsazení, kterým se v Pythonu identifikují bloky kódu, tj. neexistuje zde \verb|begin,end| jako v Pascalu nebo \verb|{}| jako v C/C++.

Více argumentů se odděluje čárkou a argumenty mohou mít výchozí hodnotu, např. argument $c$ bude 1, pokud není uvedeno jinak.
\begin{lstlisting}[caption=Definice funkce]
    def calculate(a, b, c=1):
        """
        A function that calculates c*(a + b)

        Parameters:
        -----------
        a, b, c: numbers
            arguments of the calculation

        Returns:
        --------
        result: number
            The result of calculating c*(a + b)
        """
        return c*(a + b)

    >>> calculate(2, 3)
    5
    >>> calculate(2, 3, 4)
    20
\end{lstlisting}
Zde je dlouhý řetězec ohraničený trojitými uvozovkami \lstinline{"""} takzvaný dokumentační řetězec funkce (nebo zkráceně \textbf{docstring}). Trojité uvozovky lze použít kdekoli a znamenají pouze to, že řetězec se rozkládá na více řádků. K docstringu lze poté přistupovat pomocí funkce \lstinline{help()} nebo pomocí textových editorů. Formátování použité ve výše uvedeném příkladu je běžně srozumitelné pro IDE, např. Spyder jej zobrazí pěkně naformátovaný v panelu Nápověda (standardně v pravém horním rohu okna).

Argumenty s výchozími hodnotami musí následovat za povinnými argumenty. S více volitelnými argumenty můžeme funkci volat pomocí \emph{pojmenovaných argumentů} (keyword arguments):
\begin{lstlisting}[caption=Volitelné a pojmenované argumenty.]
    def calculate2(a, b, c=1, d=1):
        return c*(a + b)**d

    >>> calculate(2, 3, d=2) # necha vychozi hodnotu c, ale nastavi d
    25
\end{lstlisting}

Funkce lze předávat v proměnných a argumentech, konkrétně funkce mohou přijímat jiné funkce jako argumenty, např.:
\begin{lstlisting}[caption=Funkce vyššího řádu.]
    def add1(x):
        return x + 1

    def change_number(x, func):
        return func(x)

    >>> change_number(1, add1)
    2
\end{lstlisting}

Krátké, jednoduché funkce (které se vejdou na jeden řádek) lze definovat přímo jako takzvané anonymní neboli \emph{lambda} funkce, např.
\begin{lstlisting}[caption=Lambda funkce.]
    >>> change_number(1, lambda x: 2 + x)
    3
\end{lstlisting}
Lambda funkce jsou obzvláště užitečné pro částečné vyplnění seznamů argumentů existujících funkcí, např.
\begin{lstlisting}
    >>> change_number(1, lambda x: calculate2(x, 4, c=4, d=0.5))
    8.94427190999916
\end{lstlisting}

\subsection{Složené typy}

\subsubsection{Seznami a n-tice (Lists and tuples)}
Seznam (\textbf{list}) je kolekce jakýchkoli pythonovských objektů definovaná pomocí hranatých závorek \verb|[]|,
\begin{lstlisting}
    my_list = [1, 2.17, "a string", ['another', 'list']]
\end{lstlisting}
K hodnotám v seznamu lze přistupovat pomocí indexování seznamu \textbf{začínajícího od 0}, např.\\\verb|my_list[1] == 2.17|. Záporné indexy se počítají od konce, tj. \verb|my_list[-1]| je poslední prvek,\\\verb|my_list[-2]| předposlední atd.

Seznamy jsou příkladem \emph{objektu}, objekty mají přidružené funkce zvané metody, které se volají pomocí tečkové notace. Užitečné metody pro seznamy jsou:
\begin{itemize}
    \item \verb|append| and \verb|pop|
\begin{lstlisting}
>>> my_list = [1, 2.0, 'three']
>>> my_list.append('one more')
>>> my_list.pop() # odstrani posledni prvek a vrati ho
"one more"
\end{lstlisting}
    \item \verb|my_list.sort()|, seřadí seznam na místě, pokud lze všechny prvky seznamu porovnat; pro vytvoření nového seznamu s seřazenými prvky můžeme použít \verb|sorted(my_list)|
    \item \verb|my_list.index('a')| najde index prvního výskytu 'a' nebo vrátí chybu
\end{itemize}

Kromě jednoduchého indexování lze k podmnožinám seznamů přistupovat pomocí \emph{řezů} (slicing), např.:
\begin{lstlisting}
    l = [1,2,3,4,5,6,7,8,9]
    l[2:5] # [3,4,5]
    l[-5::2] # [5,7,9]
\end{lstlisting}
Obecně je syntaxe \verb|my_list[start_idx:stop_idx:step]|, \verb|start_idx| je včetně, \verb|stop_idx| je vyjma. Všechny tři jsou volitelné, výchozí hodnota start je 0, výchozí hodnota stop je -1 a výchozí hodnota step je 1.

N-tice (\textbf{Tuples}) se v mnoha ohledech chovají podobně jako seznamy, s výjimkou, že jsou \textbf{neměnné} (immutable), tj. samotnou n-tici ani hodnoty v ní nelze měnit. Indexování a řezy však fungují stejně. N-tice se definují pomocí kulatých závorek \ls{()}. Pokud n-tice obsahuje pouze jednu hodnotu, je třeba přidat čárku, aby se odlišila od výrazu v závorkách, tj. \ls{single_tuple = (1,)}.

Běžným použitím n-tic je vracení více hodnot z funkce, tj.:
\begin{lstlisting}
    def add_and_subtract(x, y):
        return x+y, x-y # Vsimnete si, ze zavorky () nejsou potreba v nekterych pripadech

    # tu rozbalime vracenou dvojici do dvou promennych
    a, s = add_and_subtract(3, 2) # a==5, s==1
\end{lstlisting}
Všimněte si, že jsme neuložili návratovou hodnotu jako n-tici, ale okamžitě jsme ji rozdělili do dvou samostatných proměnných. Tomu se říká \textbf{rozbalování} (unpacking) a lze jej použít pro jakýkoli složený typ.

\subsubsection{Slovníky (Dictionaries)}
Slovníky (Dictionaries) jsou kolekce přiřazení klíče k hodnotě libovolného typu, často jsou klíče řetězce, např.:
\begin{lstlisting}
    my_dict = {
        'key1': value1,
        'key2': value2,
    }
\end{lstlisting}
which can then be accessed as \ls{my_dict['key2']} etc.

\subsection{Kontrola toku programu}

\subsubsection{if-elif-else}
Příkaz \verb|if| přijímá booleovský výraz a spustí kód ve svém bloku, pokud se podmínka vyhodnotí jako \verb|True|.
\begin{lstlisting}[language=Python]
    if condition:
        do_if_true()
    elif condition2: # nepovinne
        do_if_2true() # spusti se jenom kdyz condition je False a condition2 je True
    else: # nepovinne
        do_if_false() # spusti se jenom kdyz oboje condition and condition2 jsou False
        
\end{lstlisting}

\subsubsection{match}
Obecnější přiřazení hodnoty k akci je k dispozici pomocí \ls{match}\footnote{Dostupné od Pythonu 3.10}. 
\begin{lstlisting}
    command = 'yell'
    match command:
        case 'talk':
            print("hello")
        case 'yell':
            print("HELLO!!")
        case _: # vychozi chovini
            print("neznamy prikaz")
\end{lstlisting}

\ls{match} může být obzvláště silný v kombinaci s rozbalováním (unpacking), kde se \ls{match} snaží co nejlépe najít vzoru svého argumentu. To vám umožňuje například zpracovávat příkazy a jejich argumenty, např.:
\begin{lstlisting}
def process_command(command):
    match command:
        case ('speak', value):
            print(f"Hello {value}")
        case 'yell':
            print('HELLO!!')
        case ('yell', value):
            print(f"HELLO!! {value}")
        case (command, value):
            print(f"I'm supposed to \"{command}\" with \"{value}\")
\end{lstlisting}
což se chová jako
\begin{lstlisting}
    >>> process_command(('speak', 1))
    Hello 1
    >>> process_command('yell')
    HELLO!!
    >>> process_command(('yell', 2))
    HELLO!! 2
    >>> process_command(('shout', 20))
    I'm supposed to \"shout\" with \"20\"
\end{lstlisting}

Všimněte si však, že pokud argument pro \ls{match} neodpovídá žádnému poskytnutému vzoru, nic se nestane, tj. chová se podobně jako \ls{if} s chybějící klauzulí \ls{else}:
\begin{lstlisting}
    >>> process_command('shout')
    # zadny vystup
\end{lstlisting}

\begin{exercise}
Napište funkci, která na vstupu přijme koeficienty $a$, $b$, $c$ kvadratické rovnice $ax^2 + bx + c = 0$ a vypíše řesení $x_1$, $x_2$. 
Pokud reálné řešení neexistuje, funkce do příkazové řádky vypíše chybu.
\end{exercise}

\subsection{Smyčky (Loops)}
Pro cyklení s daným počtem iterací můžeme použít cykly \verb|for| s funkcí \verb|range()|. Můžeme také cyklit přes cokoli iterovatelného, např. seznam nebo slovník.
\begin{lstlisting}
    for k in range(10):
        print(k)

    for value in my_list:
        print(value) 
\end{lstlisting}
Obecný tvar \ls{range()} je \ls{range(start, stop, step)}, start je včetně, stop je vyjma a krok je standardně 1. \ls{range()} nevrací seznam, ale něco, co se nazývá \emph{iterátor}. Můžeme jej převést na seznam pomocí \ls{list(range(start, stop, step))}.

Jestliže potřebujeme iterovat přes nějaké hodnoty a zároveň pracovat s indexem můžeme použít \lstinline{enumerate}.
\begin{lstlisting}
    for index, value in enumerate(my_list):
        print(f"my_list[{index}] = {value}")
\end{lstlisting}

Pro cyklení, dokud je podmínka pravdivá, můžeme použít cyklus \verb|while|,
\begin{lstlisting}
    while condition:
        do_stuff()
\end{lstlisting}

Běh cyklů lze upravit pomocí \ls{continue}, které přeskočí zbytek kódu v iteraci a přejde na další, nebo \ls{break}, které cyklus okamžitě ukončí. Běžné použití je s nekonečnými cykly, např.:
\begin{lstlisting}
    while True: # bude bezet do nekonecna
        do_stuff()
        if should_we_stop: # pokud to neukoncime
            break
\end{lstlisting}

Cykly se často používají k vytváření seznamů, což lze zjednodušit pomocí \textbf{list comprehension} (generování seznamu), jako je
\begin{lstlisting}
    [expression(k) for k in iterable if condition(k)] # if podminka je nepovinna
\end{lstlisting}
For example,
\begin{lstlisting}[caption=List comprehension.]
    [k**2 for k in range(5)] # seznam druhych mocnin pro k < 5
    [k**2 for k in range(5) if k % 2 == 0] # seznam druhych mocnin pro sude k
\end{lstlisting}

\begin{exercise}
Napište program, který se zeptá uživatele na jméno a odpoví \verb|"Hello <jméno>!"|.
\begin{enumerate}
    \item Změňte program tak, aby pokud je jméno uživatele "Andrej", pozdrav se změnil na "Ciao Andrej!".
    \item Oddělte logiku vytváření pozdravu do samostatné funkce.
\end{enumerate}%
\end{exercise}

\begin{exercise}
    Napište program, který přijme libovolný počet jmen z příkazového řádku a na konci je vypíše v abecedním pořadí.
    \begin{enumerate}
        \item Uživatel předem zadá počet jmen, která chce zadat. \emph{Nápověda}: \verb|range()| a cyklus \verb|for|
        \item Program přestane žádat o nová jména, jakmile uživatel zadá prázdný řetězec. \emph{Nápověda:} \verb|break|
        \item Pokud uživatel zadal jméno "Emil", bude ve výsledném výpisu přeskočeno. \emph{Nápověda:} \verb|continue|.
    \end{enumerate}
\end{exercise}

\subsection{Používaní externích modulů}
Python je známý tím, že má velké množství dostupných knihoven pro téměř cokoli pod sluncem.\footnote{A brzy se naučíme, jak si vytvořit vlastní.} Abychom je mohli použít, musíme nejprve importovat \emph{modul}, který potřebujeme, např. pro použití funkcí, které pracují s časem, můžeme udělat
\begin{lstlisting}
    import time as t #importuj modul time a dej mu skraceny nazev t
    print(t.time()) #nazev modulu pripajime z leva ke jmenu funkce
\end{lstlisting}
Pokud chceme z daného modulu použít jen malý počet funkcí, můžeme je importovat konkrétně a nemusíme pro jejich volání používat název modulu, např.
\begin{lstlisting}
    from module import func1, func2
    func1()
    func2()
\end{lstlisting}

\begin{exercise}
    Napište program pro výpočet Fibonacciho čísla $F_n$ pomocí cyklů i rekurze. Změřte čas potřebný k výpočtu prvních 20 $F_n$ oběma metodami pomocí \verb|time.time()|.
    \emph{Nápověda:} $F_k = F_{k-1} + F_{k-2},\;F_1 = F_2 = 1$
\end{exercise}

\subsection{Objektově orientované programování}
    (OOP) je programovací technika pro spojování dat s funkcemi a oddělování implementačních detailů dílčích problémů od zbytku kódu. Při správném použití může pomoci psát snadno čitelný, rozšiřitelný a znovupoužitelný kód.

    V jazyce OOP jsou objekty instancemi tříd. Všechno v Pythonu je objekt. Například číslo 5 je instance třídy \ls{int}. V moderním Pythonu (tj. verze 3 a vyšší) mají pojmy typ a třída stejný význam. Vestavěná funkce \ls{type()} vrací třídu (nebo typ) objektu, např.
\begin{lstlisting}
    >>> type(2)
    <class 'int'>
\end{lstlisting}

    Pro vytvoření nových tříd používáme klíčové slovo \ls{class}, např. pro vytvoření třídy, která reprezentuje rozdíly
\begin{lstlisting}
class Difference:
    def __init__(self, x, y):
        self.x = x
        self.y = y
        
    def __str__(self):
        return f"{self.x} - {self.y}"
    
    def value(self):
        return self.x - self.y
\end{lstlisting}
    Funkce definované uvnitř tříd se nazývají \textbf{metody}. První argument, konvenčně nazývaný \ls{self}, odkazuje na objekt, který metodu volá. \ls{__init__} je speciální metoda, která vytváří objekt, \ls{__str__} je speciální funkce, která by měla vytvořit čitelnou textovou reprezentaci objektu (používá se s \ls{print}).

    Používání tříd a objektů je přímočaré, např.
\begin{lstlisting}
    #__init__() se zavolá s x=3, y=5 a self odkazuje na d1
    d1 = Difference(3, 5)
    d2 = Difference(30, 50) 

    print(d1.value()) # -2
    print(d2) # "30 - 50"
\end{lstlisting}

Existuje několik dalších speciálních názvů metod (více viz \href{https://docs.python.org/3/reference/datamodel.html#emulating-numeric-types}{dokumentace}),
např. \ls{__add__(self, other)}, která se volá pro \ls{x + y}, kde \ls{x} je \ls{self} a \ls{y} je \ls{other}. Podobně existují \ls{__sub__}, \ls{__mul__} a \ls{__truediv__} pro \verb|-, *, /|.

\begin{exercise}
    Implementujte třídu \ls{Fraction}, která reprezentuje zlomek, který je inicializován dvěma čísly -- čitatelem a jmenovatelem. Třída by měla podporovat základní aritmetiku (\ls{+,-,*,/}) s čísly a dalšími \ls{Fraction}.
\end{exercise}